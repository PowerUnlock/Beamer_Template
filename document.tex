\documentclass{beamer}

%\documentclass[envcountsect]{beamer}


\usepackage[english]{babel}
\usefonttheme[onlymath]{serif}

\usetheme{Hyderabad}

\usepackage{color}
\usepackage{graphicx}
\usepackage{amssymb,amsmath}
\usepackage{tikz}
\usepackage{epstopdf}
\graphicspath{{./../Figures/}}
\DeclareGraphicsExtensions{.jpeg,.png,.jpg,.eps,.pdf}

% My original color definition file

\definecolor{verm}{rgb}{0.6,0.2,0.2}
\definecolor{purp}{rgb}{0.3,0.1,0.6}
\definecolor{purple}{rgb}{0.4,0.0,0.6}
\definecolor{bggreen}{rgb}{0.1,0.3,0.1}
\definecolor{dgreen}{rgb}{0.1,0.6,0.1}
\definecolor{black}{rgb}{0.0,0.0,0.0}
\definecolor{crim}{rgb}{0.3,0.1,0.1}
\definecolor{dred}{rgb}{0.5,0.1,0.1}

% Borrowed from Alain Bensoussan

\definecolor{Blue}{cmyk}{0.65,0.13,0,0}
\definecolor{Black}{cmyk}{0,0,0,1}
\definecolor{Red}{cmyk}{0,1,1,0}
\definecolor{Green}{cmyk}{1,0,1,0}
\definecolor{Orange}{cmyk}{0,0.61,0.87,0.1}
\definecolor{Fuchsia}{cmyk}{0.47,0.91,0,0.08}
\definecolor{PineGreen}{cmyk}{0.92,0,0.59,0.25}


\input{notation.tex}

\def\Mh{\hat{M}}
\def\vbh{\hat{\v}}

% \newcommand{\btp}{\begin{tikzpicture}}
	% \newcommand{\etp}{\end{tikzpicture}}

\setlength{\parskip}{2mm}

\setbeamertemplate{theorems}[numbered]

\def\ft{\frametitle}
\def\bfr{\begin{frame}}
	\def\efr{\end{frame}}

% Notation specifically for this file.
\def\m{\tilde{m}}
\def\trm{Tr_1^{\bar{m}}}
\def\trl{Tr_1^{l}}


\def\bu{\bullet}

\AtBeginSection[]
{
	\begin{frame}<beamer>[allowframebreaks]{Outline}
		\tableofcontents[currentsection={1-2},currentsubsection]
		\framebreak
		\tableofcontents[currentsection={3-4},currentsubsection]
	\end{frame}
}

%\AtBeginSection[]
%{
%	\begin{frame}<beamer>{Outline}
%		\tableofcontents[currentsection,currentsubsection]
%	\end{frame}
%}
%
%\AtBeginSubsection[]
%{
%	  \begin{frame}<beamer>{Outline}
%		    \tableofcontents[currentsection,currentsubsection]
%		  \end{frame}
%	}

\logo{\includegraphics[height=0.5cm]{IITH-logo.jpg}}

%%%%%%%%%%%%%%%%%%%%%%%%%%%%%%%%%%% new
\makeatletter
\setbeamertemplate{footline}{%
	\leavevmode%
	\hbox{%
		\begin{beamercolorbox}[wd=.30\paperwidth,ht=2.25ex,dp=1ex,center]{author in head/foot}%
			\usebeamerfont{author in head/foot}\insertshortauthor%~~\beamer@ifempty{\insertshortinstitute}{}{(\insertshortinstitute)}
		\end{beamercolorbox}%
		\begin{beamercolorbox}[wd=.70\paperwidth,ht=2.25ex,dp=1ex,center]{title in head/foot}%
			\usebeamerfont{title in head/foot}\insertshorttitle
	\end{beamercolorbox}}%
	\vskip0pt%
}
\makeatother

\addtobeamertemplate{navigation symbols}{}{%
	\usebeamerfont{footline}%
	\usebeamercolor[fg]{footline}%
	\hspace{1em}%
	\insertframenumber/\inserttotalframenumber
}

%\expandafter\def\expandafter\insertshorttitle\expandafter{%
	%	\insertshorttitle\hfill%
	%	\insertframenumber\,/\,\inserttotalframenumber}

\begin{document}
	\title
	{  
		Deterministic Construction of Binary and Bipolar
		Measurement Matrices Using BCH Codes
	}
	%\author[]{Shashank Ranjan}
	%\institute{Systems and Control\\
		%	Indian Institute of Technology Hyderabad\\
		%}
	\author[S. Ranjan, M. Vidyasagar FRS]{Shashank Ranjan\inst{1} \and Dr. M. Vidyasagar FRS \inst{2}}
	\institute {\inst{1} PhD Scholar, IIT Hyderabad \and %
		\inst{2} SERB National Science Chair and Distinguished Professor, IIT Hyderabad}
	\date{\today}
	
	\maketitle
	
	% \bfr{Outline}
	
	% \tableofcontents
	
	% \end{frame}


\section{Introduction}
\subsection{Overview}
\bfr

Compressed sensing refers to the recovery of high-dimensional but
low-complexity data from a small number of linear measurements.

Two common applications

\begin{itemize}
	\item Vector Recovery: Reconstruct a high-dimensional but sparse
	(or nearly sparse) vector.
	\item Matrix Recovery: Reconstruct a high-dimensional but low rank
	matrix.
\end{itemize}


\end{frame}

\subsection{Problem formulation for vector recovery}
\bfr
\begin{itemize}
% \item In Compressed Sensing we have a sparse vector $x\in\R^n$.
\item The objective is to recover a sparse or a nearly sparse vector $x \in \R^n$ 
from a lower-dimentional measurment vector
\bd
y=Ax+\eta ,
\ed 
where $A\in\Rmn \ (m<n)$ is called the measurement matrix,
and $\eta\in\R^m$ is the measurement noise with 
a known upper bound $\nmeu{\eta}\leq \e$.
\item In order to recover $x$, we need a decoder map 
$\Delta : \R^{m}\rightarrow \R^{n}$.
\end{itemize}
\end{frame}

\begin{frame}

\begin{itemize}

\item Let $\SI_k \seq \R^n$ denote the set of $k$-sparse vectors, that is,
the set of vectors with no more than $k$ nonzero components.

Given a norm $\nmm{\cdot}$, let
\bd
\s_k(x, \nmm{\cdot}) := \min_{z \in \SI_k} \nmm{x-z}
\ed
denote the $k$-sparsity index of $x$, which measures how close $x$ is to
being sparse.

\item The decoder map $(\D)$ which is widely used in the 
recovery of $x$ is the \textbf{Basis pursuit} decoder map \textbf{$(\D_{BP})$} i.e.,
\be\label{eq:1}
\DBP(y) := \argmin_z \nmm{z}_1 \st \nmeu{y - Az} \leq \e .
\ee
\end{itemize}



\end{frame}

\begin{frame}

\begin{definition}\label{def:1}
The pair $(A, \DBP)$ is said achieve \textbf{robust sparse recovery} 
of order $k$ if, there exists constants $C$ and $D$ such that for all 
$\eta\in\R^m$ with $\nmeu{\eta}\leq \e$, the following inequality holds
\be\label{eq:2}
\nmm{ \DBP(Ax + \eta) - x}_p \leq C \s_k( x , \nmm{\cdot}_1 )
+ D \e , \fa x \in \R^n .
\ee
\end{definition}
The usual choices for $p$ in \eqref{eq:2} are $p = 1$ and $p = 2$.

There are two widely used sufficient 
conditions on the matrix $A$ that enable robust sparse recovery; namely,
\textbf{Restricted Isometry Property (RIP)} and
\textbf{Robust Null Space Property (RNSP)}.
\end{frame}

\subsection{Definitions of RIP and RSNP}

\bfr

\begin{definition}\label{def:2}
A matrix $A\in\Rmn$ is said to satisfy the \textbf{restricted isometry 
	property (RIP)} of order $k$ with constant $\d_k\in(0,1)$ if
\be\label{eq:3}
(1 - \d_k) \nmeusq {u} \leq \nmeusq { Au} \leq (1 + \d_k) \nmeusq {u} ,
\fa u \in \SI_k .
\ee
\end{definition}

\begin{definition}\label{def:3}
A matrix $A\in\Rmn$ is said to satisfy \textbf{robust null space property (RNSP)} 
of order $k$ with constants $\r\in(0,1)$ and $\t>0$ if, for all $S\seq[n]$ with 
$|S|\leq k$, it is the case that
\be\label{eq:4}
\nmm{h_S}_1 \leq \r \nmm{h_{S^c} }_1 + \t \nmeu{Ah} , \fa h \in \R^n .
\ee
\end{definition}

\end{frame}

\subsection{Known result}
\bfr   
Ranjan and Vidyasagar, IEEE T-SP, 2019: RIP implies RNSP. 
\begin{theorem}\label{thm:4}
Suppose $A\in\Rmn$ satisfies $RIP$ of 
order $tk$ with $\d_{tk} < \sqrt{(t-1)/t}$ for some $t>1$. Define
\bd
v:=\sqrt{t(t-1)}-(t-1)\in (0,0.5)
\ed 
\bd
a:=[v(1-v)-\d(0.5-v+v^2)]^{1/2},
\ed
\bd
b:=v(1-v)\sqrt{1+\d},\ c:=\Big[\frac{\d v^2}{2(t-1)}\Big]^{1/2}.
\ed
Then $A$ satisfies robust null space property of order $k$ with 
\bd
\r:=\frac{c}{a},\ \t:=\sqrt{k}\frac{b}{a^2}.
\ed
\end{theorem}
\end{frame}

\begin{frame}
% $RIP$ implies $RNSP$ \cite[Theorem 9]{my}
\begin{itemize}

\item This finding lead to the conclusion that RNSP is a weaker condition than RIP.

\item Subsequent to this result, in a recent work, Lotfi and Vidyasagar
have proposed a class of binary matrices that directly satisfy RNSP
while bypassing RIP.

\item They have also shown that the value of $k$ for which
the RNSP is guaranteed to hold is $3 \sqrt{3}/2 \approx 2.6$ times larger
than for the RIP.
\end{itemize}


\end{frame}

\subsection{Abstract of our work}
\bfr
\bit
\item We present a deterministic construction method for
designing the binary and bipolar compressed sensing 
matrices based on BCH codes.

\item  In our work, we adopt the methodology used
in the construction of binary matrices proposed by
Lotfi and Vidyasagar. Therefore, the matrices proposed
here directly satisfy the RNSP while
bypassing the RIP.

\item For the matrices proposed here, 
the value of $k$ for which the RNSP is guaranteed to hold is 
at least $1.3$ times larger than for the RIP.



\eit 
\end{frame}

\bfr

\bit

\item If $k$ is of the form $2^i$, where $i\in\mathbb{N}$, then
for below two cases
\begin{multline}\label{eq:5}
\frac{1}{2}\Bigl(\sqrt[3]{n+\sqrt{n^2-1/27}}+ \sqrt[3]{n-\sqrt{n^2-1/27}}\,\Bigr) \\
\leq k < \frac{1}{8} \Bigl(\sqrt{n}+\sqrt{n+16(\sqrt{n}+1)}\,\Bigr)
\end{multline}


\be\label{eq:6}
k \leq \frac{\sqrt[3]{n}}{2.83},
\ee
the matrices proposed by us require fewer 
number of measurements than the binary matrices proposed
by Lotfi and Vidyasagar.
\eit
\end{frame}

\section{BCH Codes}
\subsection{An overview of linear codes}

\bfr

\bit

\item A binary inear code denoted by $[\tilde{n},\tilde{k},d_{min}]_2$ 
is a linear subspace $\C \seq \Fbb_2^{\tilde{n}}$ of 
dimension $\tilde{k}$, where $\Fbb_2$ denotes the binary field,
and $\tilde{n}$ is the code length.


\item $|\C| = 2^{\tilde{k}}$

\item The elements of $\C$ are called codewords.

\item The \textbf{weight} of a codeword is the number of its 
non-zero elements.

\item The \textbf{distance} $d_{min}$ of a linear code 
is the minimum weight of its non-zero codewords.

\item  if vector $\oneb_m$ 
consisting of all ones is a codeword,
then $\C$ is said to be \textbf{symmetric}.

\item $\C$ is said to be cyclic if for every codeword,
its circular shif is also a valid codeword.

\eit

\end{frame}

\subsection{Narrow-sense primitive binary BCH codes}

\bfr
\begin{itemize}
	
\item Narrow-sense primitive binary BCH codes are a class of cyclic 
binary codes with $\tilde{n}= 2^{\m} -1$ ( $\m$ is a positive integer) 
which are produced by a generating polynomial $g(x)\in GF(2)[x]$                          
such that $g(x)\mid  x^{2^{\m}-1}+1$.



\item In our work, when we say that a BCH code has the designed
distance $d$, we mean that 
$\al^1,\ \al^2,\ \dotsc ,\al^{d-1}$ are different non-repeating roots of $g(x)$
( not necessarily all the roots), and $(x+1)\nmid g(x)$, where $\al\in GF(2^{\m})$ 
is a primitive root of the field.

\item Here, $d_{min}\geq d$. Since $(x+1)\nmid g(x)$, the code
will be symmetric. Which means for every codeword its complement
will aslo be a valid codeword. 

\item Since, $\tilde{n}$ is odd, each complement couple will have
one even parity codeword and one odd parity codeword.
\end{itemize}                        
\end{frame}

\section{An earlier construction based on BCH codes}

\bfr

\textbf{Amini and Marvasti, IEEE T-IT, 2011}

\begin{theorem}\label{thm:5}
Let $\C:= [\tilde{n},\ \tilde{k}, d_{min}]_2$ be a symmetric code.
Let $B\in \{0,1\}^{\tilde{n}\times 2^{\tilde{k}-1}}$ be the matrix composed
of code vectors as its column such that such that from each complement couple,
exactly one is selected. Define
\be\label{eq:7}
A = \frac{1}{\sqrt{\tilde{n}}}\Big(2 B - \oneb_{\tilde{n}\times 2^{\tilde{k}-1}}\Big).
\ee
Then $A$ satisfies the RIP with constant $\d_k = (k-1)\ (1-2\frac{d_{min}}{\tilde{n}})$
for $k< \frac{\tilde{n}}{\tilde{n}-2 d_{min}} + 1$ ( $k$ is the RIP order).
\end{theorem}

The authors used BCH code in their construction.

\end{frame}

\section{New construction method utilizing the known weight distribution of BCH codes}
\bfr

\begin{itemize}
	\item
\end{itemize}
\end{frame}
\end{document}